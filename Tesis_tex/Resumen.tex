\chapter*{Resumen}

Los robots de servicio integrados en entornos industriales surgen de la necesidad de diversificar las actividades que es posible realizar en una planta de producción y se considera benéfica la integración de visión y la percepción de profundidad en su desarrollo. Una más de las ventajas atribuidas a este tipo de robots es la reducción de riesgos al realizar labores peligrosas o que involcuren el contacto con maquinaria que pudiera resultar riesgosa para operadores humanos. Para esto es necesario que el robot pueda operar de la forma más autónoma posible, navegar en el espacio en que se encuentra e interactuar con sus alrededores de forma segura. 

En el presente documento se describe el proyecto realizado para integrar un sistema de visión computacional en el comportamiento del un robot de servicio industrial.
Para lo anterior, se integró una cámara RGBD en la estructura de un robot de servicio industrial y utilizando técnicas de visión computacional se realizó el procesamiento de los datos obtenidos. Integrando esta información en una secuencia de acciones que permiten al robot manipular un objeto determinado dentro de su espacio de trabajo y transportarlo a otra locación.
Los resultados se integraron dentro de los lineamientos establecidos por la competencia \textit{RoboCup Logistics}, donde se busca simular un sistema de producción en serie. Dentro de este entorno es posible hacer pruebas para el desarrollo de robots de servicio para entornos industriales.