\chapter*{Abstract}
The integration of service robots in industrial environments comes from the urge to diversify the activities that can be made in a production plant. As ground for this project it is considered benefical to integrate computer vision and depth perception in its behaviour. One of the advantages of these robots is the reduction the risks, by performing dangerous activities or those that involve contact with machines that represent a risk to human operators. To complain with this objective it is necessary that te robot can operate in the most autonomous way possible, navigating through the environment and interact with its surroundings in a safe way.

This document describe a project made to integrate a computer vision system in an industrial service robot's behaviour. In order to achieve it, a RGBD camera was added to the hardware of a mobile robot, this information was processed later using computer vision techniques. The processed information was used during a sequence of actions, organized using Finite State Machines, that dictate the robot's behaviour, with which it is able to manipulate an object in its workspace and transport it to another location.
The results were evaluated following the guidelines by the \textit{RoboCup Logistics} competition, where a simulation of a production system takes place. In this setting it is possible to make tests to develop industrial service robots.