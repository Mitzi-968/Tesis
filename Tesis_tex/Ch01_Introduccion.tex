\chapter{Introducción}
\pagenumbering{arabic}

 Actualmente uno de los temas que se encuentran en apogeo y con mucho reconocimiento de la aplicación de los avances tecnológicos es la robótica. Constantemente se desarrollan nuevas tecnologías que es posible implementar tanto en el software como en el hardware de estas herramientas, además de la paulatina integración de estos elementos en la vida diaria, en entornos y aplicaciones diversas.


 No resulta extraño mencionar que el ambiente donde con más frecuencia se encuentran estos elementos es el industrial, donde desde hace varias décadas es posible encontrar centros de producción masiva en que brazos robóticos y bandas transportadoras se encargan del ensamblaje en serie de diferentes productos. Utilizando dicho modelo, donde se encuentran comunicados y \textit{encadenadas} todas las etapas de la producción de los productos, se obtienen fábricas en las que es posible producir una gran cantidad del mismo tipo de producto.


 El anterior es uno de los motivos por los cuales se ha encontrado necesaria la introducción de modelos de producción más flexibles, quizás para plantas menos especializadas o para productos que no requieran ser construidos de forma masiva, haciendo posible utilizar las alas industriales de formas variadas para el ensamblaje de productos variados. Así, considerando que las máquinas que se encargan de las etapas de producción no se encuentren físicamente ligadas unas con otras, aunque sí se encarguen de desempeñar tareas específicas dentro de un proceso secuencial, se obtiene el concepto de fábricas inteligentes, correspondientes a la llamada Industria 4.0, aveces mencionada como la 4ta revolución industrial \cite{kohout_multi-robot_2020}.   

El Instituto Norteamericano de Robótica, describe el \textbf{robot industrial} como: \textit{un manipulador multifuncional y programable, diseñado para mover materiales, piezas, herramientas o dispositivos especiales mediante movimientos programados y variables que permiten llevar a cabo diversas tareas.}\cite{navarro_pina_robot_2021}, siendo estos los antes mencionados brazos armadores que típicamente se ven en las plantas industriales. 

Por otro lado, Basco et al., 2018 \cite{basco_industria_2018}, enfatizan la importancia de la robótica colaborativa, en la cual la flexibilidad y versatilidad de las plantas es imperativa, integrando sistemas que puedan sumarse al sistema de producción  mediante el transporte de productos intermedios y finales entre estaciones de trabajo, que sean capaces de coexistir y colaborar con equipos similares a ellos o bien con trabajadores humanos. Siendo este último el tipo de robots con el que se plantea el desarrollo del presente proyecto.
 
\newpage
\section{Motivación}
El acelerado avance de la tecnología y las necesidades de la sociedad vuelven indispensable la búsqueda de herramientas que permitan mejorar las capacidades de los robots para analizar e interactuar con sus alrededores. Dotar al robot con las herramientas necesarias para adaptarse a su entorno de trabajo es una tarea crucial para muchas aplicaciones, tanto en el entorno industrial, como en el doméstico, en labores de ensamblaje o en tareas de interacción humano-robot \cite{roveda_robot_2022}. Con dicho constante avance tecnológico ha aumentado también el aprovechamiento de los sistemas de visión con los que es posible  equipar a los robots, dando lugar al incremento de la autonomía de los mismos y expandiendo la posibilidad de explorar su entorno de trabajo con nuevas y más poderosas herramientas y sensores, mejorando así su desempeño.


Muchas aplicaciones pueden beneficiarse de la visión computacional para mejorar la realización de una actividad, ejemplo de estas pueden ser tareas de ensamblado que requieran alta precisión, colaboración entre robots y humanos, operaciones de robots móviles, entre otras \cite{ROVEDA2021103711}. Es por este motivo que el desarrollo de esta investigación busca trabajar con robots de servicio para fábricas inteligentes, cuyo objetivo es realizar tareas útiles para humanos o otros robots en entornos industriales. En este tipo de situaciones es importante que el robot de servicio sea capaz de planear la ruta mediante la cual navegará dentro del entorno, además de qué trayectoria de movimiento seguirá el manipulador, calculando la fuerza a ejercer en cada uno de los actuadores que se encuentren involucrados en dicho movimiento, la orientación desde la que debe aproximarse al objeto que desee manipular, etcétera \cite{Rosenbaum2006}.


Este proyecto se realizó en las instalaciones del laboratorio de Bio-robótica de la Facultad de Ingeniería, bajo la asesoría de los Doctores Jesús Savage Carmona y Marco Antonio Negrete Villanueva. La robótica de servicio ha sido el tema de principal interés entre de los proyectos desarrollados por los colaboradores del laboratorio en años anteriores, por lo que el presente es uno de los primeros trabajos en que se explora la robótica industrial. Para ello, se consideró posible extrapolar las técnicas y algoritmos explorados previamente en robótica de servicio y evaluar su desempeño en ambientes industriales. Además, aplicando estas necesidades al entorno social en que se desarrolla este proyecto, se considera conveniente permanecer actualizados con los desafíos que se presentan tanto en términos académicos como en los requerimientos de la industria, por lo que proponer el desarrollo y estudio de estos robots y la incursión en este cambio de paradigma sobre la forma en que las fábricas han funcionado tradicionalmente, puede representar una colaboración de valor para el laboratorio de Biorobótica.

Para cumplir con esta meta, se requiere una modificación de las capacidades y herramientas de los robots para ser utilizados en escenarios industriales e interactuar con los elementos y personas que los componen, es por eso que uno más de los objetivos que se persiguen es otorgar a un robot suficiente autonomía para que realice las tareas que se le asignan requiriendo la menor intervención humana posible. El avance tecnológico previamente mencionado da lugar al uso de nuevos instrumentos que permiten enriquecer la diversidad, la cantidad, la precisión y el tipo de datos que recaba el robot sobre su entorno \cite{basco_industria_2018}. Estos datos en conjunto pueden ser utilizados para mejorar el desempeño del robot.

\section{Planteamiento del problema}
La visión computacional, la planeación de acciones y la manipulación de objetos siguen siendo consideradas como problemas abiertos en el campo de la robótica. Evidencia de esto son las diversas competencias que se llevan a cabo año con año, en las cuales se evalúan, entre otras, las herramientas mencionadas. \cite{sun_research_2022}. Aunado a estas áreas de oportunidad en el desempeño de los robots, es posible integrar dispositivos que permitan recabar más información del ambiente, buscando reducir riesgos y mejorar progresivamente el conocimiento del robot sobre su entorno. En las competencias mencionadas es posible evaluar el desempeño que distintos enfoques tienen para resolver el mismo problema, y abren oportunidades de colaboración entre entidades educativas de diferentes países y permiten la discusión de problemáticas comunes entre estas instituciones.


El procedimiento propuesto se inspira en la competencia \textit{RoboCup Logistics League} que se realiza haciendo uso la plataforma educativa \textbf{Robotino} de la empresa \textbf{FESTO}, en el cual un equipo de máximo 3 robots debe trabajar de manera colaborativa en la ejecución de diferentes tareas que componen la simulación de un proceso de producción industrial \cite{robocup__mathworks_robocup_2022}. Esto requiere compartir información y navegar en un espacio común dentro del cual se encuentran estaciones de trabajo y otros robots que también se encontrarán en movimiento y posiblemente haciendo uso de las estaciones de trabajo.


Esta competencia tiene como objetivo la construcción de varios tipos de \textit{productos}, que consiste en el ensamblado de piezas. Dichos productos se pueden encontrar conformados por distintas combinaciones de cantidad y color de elementos en un orden específico. Este planteamiento hace indispensable poder identificar las características de los elementos que se encuentren en las estaciones de trabajo, y poder clasificarlas por categoría y color, así como determinar la forma óptima en que es conveniente que el robot manipule dichas piezas, por lo que la implementación de algoritmos de visión es también uno de los principales objetivos de esta tarea.

Dado que se trata de la primera incursión en esta competencia, el planteamiento para este proyecto aborda una sección limitada de la misma, donde se evalúan las habilidades aisladas que es necesario que el robot pueda realizar antes de participar en el proceso de ensamblado completo. Estas habilidades son: \textbf{navegación}, \textbf{exploración}, y \textbf{manipulación de objetos}. 

\section{Hipótesis}
Es posible aumentar el nivel de autonomía de un robot de servicio para fábricas inteligentes al poder planear una secuencia de acciones para realizar movimientos haciendo uso de los sensores disponibles logrando un análisis de su entorno, así como optimizar este análisis al implementar un sistema de visión activa que sea capaz de adaptarse a las condiciones del ambiente, mejorando el desempeño general del robot.

\section{Objetivos}
\subsection{General}
Integrar la Visión Computacional como parte del comportamiento de un robot móvil que colabore en actividades de producción industrial además de  llevar a cabo una secuencia de acciones que permita llevar a término la cadena de elaboración de un producto final.
\subsection{Específicos}
\begin{itemize}
    \item Procesar la imagen obtenida por una cámara RGB en la estructura de un robot para identificar objetos en el espacio de trabajo de acuerdo a sus características.
    \item Utilizar la nube de puntos  obtenida de la cámara de profundidad para conocer la posición relativa de los objetos y plataformas de interés y planear una trayectoria del manipulador para tomarlos. 
    \item Identificar los objetos de interés presentes dentro del espacio de trabajo de un robot de servicio para fábricas inteligentes y sujetarlos exitosamente con el manipulador del robot.
\end{itemize}
%\section{Descripción del documento}